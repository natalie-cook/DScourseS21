\documentclass[12pt,english]{article}
\usepackage{mathptmx}

\usepackage{color}
\usepackage[dvipsnames]{xcolor}
\definecolor{darkblue}{RGB}{0.,0.,139.}

\usepackage[top=1in, bottom=1in, left=1in, right=1in]{geometry}

\usepackage{amsmath}
\usepackage{amstext}
\usepackage{amssymb}
\usepackage{setspace}
\usepackage{lipsum}

\usepackage[authoryear]{natbib}
\usepackage{url}
\usepackage{booktabs}
\usepackage[flushleft]{threeparttable}
\usepackage{graphicx}
\usepackage[english]{babel}
\usepackage{pdflscape}
\usepackage[unicode=true,pdfusetitle,
 bookmarks=true,bookmarksnumbered=false,bookmarksopen=false,
 breaklinks=true,pdfborder={0 0 0},backref=false,
 colorlinks,citecolor=black,filecolor=black,
 linkcolor=black,urlcolor=black]
 {hyperref}
\usepackage[all]{hypcap} % Links point to top of image, builds on hyperref
\usepackage{breakurl}    % Allows urls to wrap, including hyperref

\linespread{2}

\begin{document}

\begin{singlespace}
\title{Returns to Language Skills in the US Labor Market\thanks{Acknowledgements here, if any.}}
\end{singlespace}

\author{Natalie Cook\thanks{Department of Economics, University of Oklahoma.\
E-mail~address:~\href{mailto:student.name@ou.edu}{natalie.c.cook-1@ou.edu}}}

% \date{\today}
\date{April 27, 2021}

\maketitle

\begin{abstract}
\begin{singlespace}
Figuring out which language gives you the biggest wage increase in the US holding as much other stuff constant as possible. Using linear regression plus lots of fancy data science techniques. This is really for fun and I don't think there are any serious implications of the findings. Maybe on a personal level people could choose which language to learn based on what will make them the most money. 
\end{singlespace}

\end{abstract}
\vfill{}


\pagebreak{}


\section{Introduction}\label{sec:intro}
This is a silly topic. I'm doing it to entertain myself and learn. There are no cancer-curing implications of it. 

\section{Literature Review}\label{sec:litreview}
Previous work by \citet{rhur} shows that educational decisions are an important determinant of later-life earnings. This point is driven further in follow-up work by \citet{altonji_al2012} and \citet{altonji_al2016}.

There is one paper that is almost exactly my topic, and then everything else is hyper specific to some other circumstance. I don't think that people actually research this seriously so it's kinda hard to do a lit review. 

\section{Data}\label{sec:data}

The primary data source for this research is the 2019 American Community Survey from the US Census Bureau. I will put a fancy table reference here to the summary statistics \ref{tab:descriptives}. 
Now I will talk about the survey in general a little bit and the go into depth on the variables I will use and how they are measured. This is important for interpretation later on. 


\section{Empirical Methods}\label{sec:methods}
The primary empirical model for the analysis can be depicted in the following equation:

\begin{equation}
\label{eq:1}
Y_{it}=\alpha_{0} + \alpha_{1}Z_{it} + \alpha_{2} X_{it} + \varepsilon,
\end{equation}
where $Y_{it}$ is a continuous outcome variable for unit $i$ in year $t$, and $Z_{it}$ are characteristics about the firm at which $i$ is working, while $X_{it}$ are characteristics about $i$. The parameter of interest is $\alpha_{1}$.

log wage is a function of language, occupation, industry, tenure/age, sex, race, location, educational attainment
I'll tell you about why I'm using logwage when I figure it out myself - does that count as data cleaning and mutation?

I think I have to make like 100,000 dummy variables because almost everything is categorical here. Y which is log wage is continuous. I think it's gonna be a big long mess.

\section{Research Findings}\label{sec:results}
The main results are reported in Table \ref{tab:estimates}.
This is where I try to interpret my results. I'll comment on whether the model overall is valid/significant. I'm hoping that it is. I'll talk about heteroskedasticity and robust standard error and other things that I definitely remember that make a model legit. Then I'll interpret the coefficients that get calculated for each variable. Maybe not each variable since some of them are supposed to be controls. 


\section{Conclusion}\label{sec:conclusion}
If you want to pick a language to learn that will increase your salary - please do an actually nuanced assessment of your situation. If you're gonna go into national security or intelligence of course you should learn Russian or Korean or Arabic or Mandarin. If you need something that is broadly applicable you should learn Spanish. This is dumb and not actually useful and you shouldn't use it as advice because I can almost guarantee that whatever it tells you to do will actually be wrong. It's a fun exercise though so there's that. Don't actually take this conclusion seriously, this is for my learning and entertainment purposes. 

\vfill
\pagebreak{}
\begin{spacing}{1.0}
\bibliographystyle{jpe}
\bibliography{References.bib}
\addcontentsline{toc}{section}{References}
\end{spacing}

\vfill
\pagebreak{}
\clearpage

%========================================
% FIGURES AND TABLES 
%========================================
\section*{Figures and Tables}\label{sec:figTables}
\addcontentsline{toc}{section}{Figures and Tables}
%----------------------------------------
% Figure 1
%----------------------------------------
\begin{figure}[ht]
\centering
\bigskip{}
\includegraphics[width=.9\linewidth]{fig1.eps}
\caption{Figure caption goes here}
\label{fig:fig1}
\end{figure}

%----------------------------------------
% Table 1
%----------------------------------------
\begin{table}[ht]
\caption{Summary Statistics of Variables of Interest}
\label{tab:descriptives} 
\centering
\begin{threeparttable}
\begin{tabular}{lcccc}
&&&&\\
\multicolumn{5}{l}{\emph{Panel A: Summary Statistics for Variables of Interest}}\\
\toprule
                                                        & Mean  & Std. Dev. & Min   & Max   \\
\midrule
Outcome variable 1                                      & 4.127 & 1.709     & 0.000 & 8.516 \\
Outcome variable 2                                      & 1.293 & 0.648     & 0.000 & 0.216 \\
Policy variable                                         & 0.685 & 0.464     & 0.000 & 1.000 \\
Control variable 1                                      & 0.451 & 0.497     & 0.000 & 1.000 \\
Control variable 2                                      & 0.322 & 0.467     & 0.000 & 1.000 \\
&&&&\\
\multicolumn{5}{l}{\emph{Panel B: Sample Means of Outcome Variables for Subgroups}}\\
\midrule
                                                        & Group 1 & Group 2 & Group 3 & Group 4 \\
\midrule
Outcome variable 1                                      & 1.782  & 2.181  & 3.749  & 4.127  \\
Outcome variable 2                                      & 0.824  & 0.971  & 1.215  & 1.693  \\
\midrule
$N$                                                     & 25,796 & 75,879 & 37,157 & 33,839 \\
\bottomrule
\end{tabular}
\footnotesize Notes: Put any notes about the table here. Sample size for all variables in Panel A is $N=172,671$.
\end{threeparttable}
\end{table}


%----------------------------------------
% Table 2
%----------------------------------------
\begin{table}[ht]
\caption{Empirical estimates of parameter of interest}
\label{tab:estimates} 
\centering
\begin{threeparttable}
\begin{tabular}{lcc}
\toprule
                            & Few Controls    & Many Controls \\
\midrule
Variable of interest        & -1.977***       & -0.536**    \\
                            & (0.219)         & (0.214)     \\
Individual characteristics  & $\checkmark$    & $\checkmark$\\
Firm characteristics        &                 & $\checkmark$\\
Location dummies            &                 & $\checkmark$\\
\midrule
$N$                         & 172,671         & 172,671      \\
\bottomrule
\end{tabular}
\footnotesize Notes: Table notes here. Standard errors in parentheses. ***Significantly different from zero at the 1\% level; **Significantly different from zero at the 5\% level.
\end{threeparttable}
\end{table}


\end{document}
